% !TeX spellcheck = en_US

% \documentclass[10pt,journal,compsoc]{IEEEtran}
% \documentclass[conference,compsoc]{IEEEtran}
\documentclass[10pt, conference, letterpaper]{IEEEtran}
% keep IEEEtran.cls at the same path


% 自定义符号
%%%%%%%%%%%%%%%%%%%%%%%%%%%%%%%%%%%%%%%%%%%%%%%%%%%%%%%%%%%%%%
%                     self defined marks
%=============================================================
% \newcommand{\mmset}[2]{\{{#1}_1,{#1}_2,\cdots,{#1}_#2\}}

% \newcommand{\mListNumber}[1]{({#1}=1,2,3,\dots)}


%=============================================================
%                       end definition
%%%%%%%%%%%%%%%%%%%%%%%%%%%%%%%%%%%%%%%%%%%%%%%%%%%%%%%%%%%%%%

% 必备包


%============packages used in this article====================
% 调整字体颜色
\usepackage{color, xcolor}

% 数学公式
\usepackage{amsmath}

% 证明
\usepackage{amsthm}
\newtheorem{theorem}{Theorem}
\newtheorem{lemma}{Lemma}
% \newtheorem{proof}{Proof}[section]



% 使用array
\usepackage{array}

% 美化表格
% http://ctan.cs.uu.nl/macros/latex/contrib/booktabs/booktabs.pdf
\usepackage{booktabs}

% 允许使用对号
\usepackage{amssymb}

% 双栏布局最后一行对齐
%\usepackage{flushend}


% 引用显示为红色
% \usepackage[colorlinks,linkcolor=red,anchorcolor=red,citecolor=red]{hyperref}
% \usepackage[numbers,sort&compress]{natbib} 


% 使用\begin{figure}[H]不让浮动体乱跑
\usepackage{float}

% 使用图片
% 图片文件路径
\usepackage[pdftex]{graphicx} 
\graphicspath{{graphics/},{graphics/authors/}}

% 子图
\usepackage{subfigure}

% *** CITATION PACKAGES ***
%
\ifCLASSOPTIONcompsoc
% IEEE Computer Society needs nocompress option
% requires cite.sty v4.0 or later (November 2003)
\usepackage[nocompress]{cite}
\else
% normal IEEE
\usepackage{cite}
\fi



% For Algorithm
\makeatletter
\newif\if@restonecol
\makeatother
\let\algorithm\relax
\let\endalgorithm\relax
\usepackage[linesnumbered,ruled,vlined]{algorithm2e} %[ruled,vlined]{
\usepackage{algpseudocode}
\renewcommand{\algorithmicrequire}{\textbf{Input:}}  % Use Input in the format of Algorithm
\renewcommand{\algorithmicensure}{\textbf{Output:}} % Use Output in the format of Algorithm


%=============================================================


% correct bad hyphenation here
% 手动调节特定单词在行尾如何断开
\hyphenation{op-tical net-works semi-conduc-tor}


\begin{document}





% //* 标题
% \title{Joint Request Dispatching and Resource Scheduling in Mobile Edge Computing}
\title{XXXXXX}


% \input{_author}









% make the title area
\maketitle




% \IEEEtitleabstractindextext{%
\begin{abstract}
The abstract goes here. 
\end{abstract}

%* IEEE COMP society keywords
% \begin{IEEEkeywords}
% Computer Society, IEEE, IEEEtran, journal, \LaTeX, paper, template.
% \end{IEEEkeywords}
% }


%* ACM sigconf keywords
\keywords{datasets, neural networks, gaze detection, text tagging} % no keywords

% For peer review papers, you can put extra information on the cover
% page as needed:
% \ifCLASSOPTIONpeerreview
% \begin{center} \bfseries EDICS Category: 3-BBND \end{center}
% \fi
%
% For peerreview papers, this IEEEtran command inserts a page break and
% creates the second title. It will be ignored for other modes.
\IEEEpeerreviewmaketitle





% \IEEEraisesectionheading{\section{Introduction}\label{sec:introduction}} % for IEEE comp society journel
\section{Introduction}\label{sec:introduction}
% \subsection{Background and Motivation}

\subsection{Main Contributions}


\begin{enumerate}
    \item   
    \item   
    \item   
\end{enumerate}

% 引用图片
% \begin{figure}
%     \begin{center}
%         \includegraphics[width=0.45\textwidth]{}
%     \end{center}
%     \caption{}
%     \label{fig:}
% \end{figure}

\input{_system_model}

\section{Conclusion}
The conclusion goes here. So what is the conclusion? You can get it soon.

\input{_related_work}

\section{Acknowledgments}
The authors would like to thank...



%================reference========================
%\scriptsize

\footnotesize
\bibliographystyle{IEEEtran}
\bibliography{reference, reference-3gpp}
%================reference========================


% that's all folks
\end{document}


