% !TeX spellcheck = en_US

% \documentclass[10pt,journal,compsoc]{IEEEtran}
% \documentclass[conference,compsoc]{IEEEtran}
\documentclass[10pt, conference, letterpaper]{IEEEtran}
% keep IEEEtran.cls at the same path


% 自定义符号
%%%%%%%%%%%%%%%%%%%%%%%%%%%%%%%%%%%%%%%%%%%%%%%%%%%%%%%%%%%%%%
%                     self defined marks
%=============================================================
% \newcommand{\mmset}[2]{\{{#1}_1,{#1}_2,\cdots,{#1}_#2\}}

% \newcommand{\mListNumber}[1]{({#1}=1,2,3,\dots)}


%=============================================================
%                       end definition
%%%%%%%%%%%%%%%%%%%%%%%%%%%%%%%%%%%%%%%%%%%%%%%%%%%%%%%%%%%%%%

% 必备包


%============packages used in this article====================
% 调整字体颜色
\usepackage{color, xcolor}

% 数学公式
\usepackage{amsmath}

% 证明
\usepackage{amsthm}
\newtheorem{theorem}{Theorem}
\newtheorem{lemma}{Lemma}
% \newtheorem{proof}{Proof}[section]



% 使用array
\usepackage{array}

% 美化表格
% http://ctan.cs.uu.nl/macros/latex/contrib/booktabs/booktabs.pdf
\usepackage{booktabs}

% 允许使用对号
\usepackage{amssymb}

% 双栏布局最后一行对齐
%\usepackage{flushend}


% 引用显示为红色
% \usepackage[colorlinks,linkcolor=red,anchorcolor=red,citecolor=red]{hyperref}
% \usepackage[numbers,sort&compress]{natbib} 


% 使用\begin{figure}[H]不让浮动体乱跑
\usepackage{float}

% 使用图片
% 图片文件路径
\usepackage[pdftex]{graphicx} 
\graphicspath{{graphics/},{graphics/authors/}}

% 子图
\usepackage{subfigure}

% *** CITATION PACKAGES ***
%
\ifCLASSOPTIONcompsoc
% IEEE Computer Society needs nocompress option
% requires cite.sty v4.0 or later (November 2003)
\usepackage[nocompress]{cite}
\else
% normal IEEE
\usepackage{cite}
\fi



% For Algorithm
\makeatletter
\newif\if@restonecol
\makeatother
\let\algorithm\relax
\let\endalgorithm\relax
\usepackage[linesnumbered,ruled,vlined]{algorithm2e} %[ruled,vlined]{
\usepackage{algpseudocode}
\renewcommand{\algorithmicrequire}{\textbf{Input:}}  % Use Input in the format of Algorithm
\renewcommand{\algorithmicensure}{\textbf{Output:}} % Use Output in the format of Algorithm


%=============================================================


% correct bad hyphenation here
% 手动调节特定单词在行尾如何断开
\hyphenation{op-tical net-works semi-conduc-tor}


\begin{document}





% //* 标题
% \title{Joint Request Dispatching and Resource Scheduling in Mobile Edge Computing}
\title{XXXXXX}


% %%
%% The "author" command and its associated commands are used to define
%% the authors and their affiliations.
%% Of note is the shared affiliation of the first two authors, and the
%% "authornote" and "authornotemark" commands
%% used to denote shared contribution to the research.
\author{Ben Trovato}
\authornote{Both authors contributed equally to this research.}
\email{trovato@corporation.com}
\orcid{1234-5678-9012}
\author{G.K.M. Tobin}
\authornotemark[1]
\email{webmaster@marysville-ohio.com}
\affiliation{%
  \institution{Institute for Clarity in Documentation}
  \streetaddress{P.O. Box 1212}
  \city{Dublin}
  \state{Ohio}
  \country{USA}
  \postcode{43017-6221}
}

\author{Lars Th{\o}rv{\"a}ld}
\affiliation{%
  \institution{The Th{\o}rv{\"a}ld Group}
  \streetaddress{1 Th{\o}rv{\"a}ld Circle}
  \city{Hekla}
  \country{Iceland}}
\email{larst@affiliation.org}

\author{Valerie B\'eranger}
\affiliation{%
  \institution{Inria Paris-Rocquencourt}
  \city{Rocquencourt}
  \country{France}
}

\author{Aparna Patel}
\affiliation{%
 \institution{Rajiv Gandhi University}
 \streetaddress{Rono-Hills}
 \city{Doimukh}
 \state{Arunachal Pradesh}
 \country{India}}

\author{Huifen Chan}
\affiliation{%
  \institution{Tsinghua University}
  \streetaddress{30 Shuangqing Rd}
  \city{Haidian Qu}
  \state{Beijing Shi}
  \country{China}}

\author{Charles Palmer}
\affiliation{%
  \institution{Palmer Research Laboratories}
  \streetaddress{8600 Datapoint Drive}
  \city{San Antonio}
  \state{Texas}
  \country{USA}
  \postcode{78229}}
\email{cpalmer@prl.com}

\author{John Smith}
\affiliation{%
  \institution{The Th{\o}rv{\"a}ld Group}
  \streetaddress{1 Th{\o}rv{\"a}ld Circle}
  \city{Hekla}
  \country{Iceland}}
\email{jsmith@affiliation.org}

\author{Julius P. Kumquat}
\affiliation{%
  \institution{The Kumquat Consortium}
  \city{New York}
  \country{USA}}
\email{jpkumquat@consortium.net}

%%
%% By default, the full list of authors will be used in the page
%% headers. Often, this list is too long, and will overlap
%% other information printed in the page headers. This command allows
%% the author to define a more concise list
%% of authors' names for this purpose.
\renewcommand{\shortauthors}{Trovato and Tobin, et al.}









% make the title area
\maketitle




% \IEEEtitleabstractindextext{%
\begin{abstract}
The abstract goes here. 
\end{abstract}

%* IEEE COMP society keywords
\begin{IEEEkeywords}
Computer Society, IEEE, IEEEtran, journal, \LaTeX, paper, template.
\end{IEEEkeywords}


 % no keywords

% For peer review papers, you can put extra information on the cover
% page as needed:
% \ifCLASSOPTIONpeerreview
% \begin{center} \bfseries EDICS Category: 3-BBND \end{center}
% \fi
%
% For peerreview papers, this IEEEtran command inserts a page break and
% creates the second title. It will be ignored for other modes.
\IEEEpeerreviewmaketitle





% *=======================规则说明=======================

% Computer Society journal (but not conference!) papers do something unusual
% with the very first section heading (almost always called "Introduction").
% They place it ABOVE the main text! IEEEtran.cls does not automatically do
% this for you, but you can achieve this effect with the provided
% \IEEEraisesectionheading{} command. Note the need to keep any \label that
% is to refer to the section immediately after \section in the above as
% \IEEEraisesectionheading puts \section within a raised box.




% The very first letter is a 2 line initial drop letter followed
% by the rest of the first word in caps (small caps for compsoc).
% 
% form to use if the first word consists of a single letter:
% \IEEEPARstart{A}{demo} file is ....
% 
% form to use if you need the single drop letter followed by
% normal text (unknown if ever used by the IEEE):
% \IEEEPARstart{A}{}demo file is ....
% 
% Some journals put the first two words in caps:
% \IEEEPARstart{T}{his demo} file is ....
% 
% Here we have the typical use of a "T" for an initial drop letter
% and "HIS" in caps to complete the first word.
% *^^^^^^^^^^^^^^^^^^^^^^^^^^^规则说明^^^^^^^^^^^^^^^^^^^^^^^^^^^





\IEEEraisesectionheading{\section{Introduction}\label{sec:introduction}}

\IEEEPARstart{T}{his} demo file is intended to serve as a ``starter file''
for IEEE Computer Society journal papers produced under \LaTeX\ using
IEEEtran.cls version 1.8b and later.
% You must have at least 2 lines in the paragraph with the drop letter
% (should never be an issue)
I wish you the best of success.


% \hfill 为 右对齐 命令
\hfill mds

\hfill August 26, 2015






\subsection{Main Contributions}


\begin{enumerate}
    \item  \gls{upf}
    \item   \gls{upf}
    \item   
\end{enumerate}

% 引用图片
% \begin{figure}
%     \begin{center}
%         \includegraphics[width=0.45\textwidth]{}
%     \end{center}
%     \caption{}
%     \label{fig:}
% \end{figure}

\input{_system_model}

\section{Conclusion}
The conclusion goes here. So what is the conclusion? You can get it soon.

\section{Related Work}


% use section* for acknowledgment
\ifCLASSOPTIONcompsoc
  % The Computer Society usually uses the plural form
  \section*{Acknowledgments}
\else
  % regular IEEE prefers the singular form
  \section*{Acknowledgment}
\fi


The authors would like to thank...



%================reference========================
%\scriptsize

\footnotesize
\bibliographystyle{IEEEtran}
\bibliography{reference, reference-3gpp}
%================reference========================


% that's all folks
\end{document}


