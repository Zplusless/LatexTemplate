

% e.g.
% \newacronym{标签}{缩写}{全称}

%使用教程 https://blog.csdn.net/zhisuihen6347/article/details/88901804
% \gls 直接引用
% \glspl 复数形式,一般直接加s,如果特殊的需要在上面\newglossaryentry的plural里面定义啦
% \Gls 首字母大写

% 一般写论文,摘要和正文是两个独立的部分,所以摘要第一次缩写出现需要有全称,正文也是的。
% 这边就需要\glsreset{}指令了,在正文开始的时候reset一下就行了,
% 比如上面的FN指令就是\glsreset{FN}。


% 如果需要打印缩略语表,在需要的地方使用\printglossaries


\newacronym{mec}{MEC}{Mobile Edge Computing}

\newacronym{af}{AF}{Application Function}

\newacronym{a2c}{A2C}{Advantage Actor Critic}

\newacronym{qos}{QoS}{Quality of Service}

\newacronym{ttl}{TTL}{Tine to Live}

\newacronym{gtpu}{GTP-U}{GPRS Tunneling Protocol User plane}

\newacronym{upf}{UPF}{User Plane Function}

\newacronym{psa}{PSA}{Protocol Data Unit Session Anchor}

\newacronym{ulcl}{UL CL}{Uplink Classifier}

\newacronym{iupf}{I-UPF}{Intermediate UPF}

\newacronym{aupf}{A-UPF}{Anchor UPF}

\newacronym{ue}{UE}{User Equipment}